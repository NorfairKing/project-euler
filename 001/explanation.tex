\documentclass[main.tex]{subfiles}
\begin{document}

\subsection{Problem 1: Multiples of 3 and 5}
\subsubsection{A Naïve solution}

An input size of $p=1000$ is small enough to let us get away with a naïve solution.
The naïve solution, to check every number, gives us an $O(p)$ runtime.

\begin{figure}[H]
\centering
\lstinputlisting[language=c, firstline=11, lastline=21]{\sol{001}{c}/naive.c}
\caption{A Naïve solution}
\end{figure}

\subsubsection{A better solution}
As it turns out, there's a better way to solve this problem.
Because the numbers that are divisible by $3$ and $5$ are easily predicitible, we can calculate the solution with the following formula:

\[
  \sum_{i=0}^{3i < p}3i
+ \sum_{i=0}^{5i < p}5i
- \sum_{i=0}^{15i < p}15i
\]

\begin{figure}[H]
\centering
\lstinputlisting[language=c, firstline=12, lastline=29]{\sol{001}{c}/better.c}
\caption{A Better solution}
\end{figure}

In this solution, we check less than half of the numbers we did in the last solution.

\subsubsection{A constant-time solution}
Looking at the above formula, we see possibilities to optimise even further.
First we have to rewrite the formula, but for the sake of simplicity we will count numbers up to \textit{and including} $p$.

\begin{align*}
& \sum_{i=0}^{3i \le p}3i
+ \sum_{i=0}^{5i \le p}5i
- \sum_{i=0}^{15i \le p}15i
\\
&= \sum_{i=0}^{i \le \floor{\frac{p}{3}}}3i
+ \sum_{i=0}^{i \le \floor{\frac{p}{5}}}5i
- \sum_{i=0}^{i \le \floor{\frac{p}{3\cdot 5}}}15i
\\
&= 3\sum_{i=0}^{\floor{\frac{p}{3}}}i
+ 5\sum_{i=0}^{\floor{\frac{p}{5}}}i
- 15\sum_{i=0}^{\floor{\frac{p}{3\cdot 5}}}i
\\
&= 3\frac{\floor{\frac{p}{3}}\cdot \left(\floor{\frac{p}{3}} + 1\right)}{2}
+ 5\frac{\floor{\frac{p}{5}}\cdot \left(\floor{\frac{p}{5}} + 1\right)}{2}
- 15\frac{\floor{\frac{p}{15}}\cdot \left(\floor{\frac{p}{15}} + 1\right)}{2}
\end{align*}

The critical insight is here the explicit formula for the sum of the first $n$ nont-zero integers:
\[
\sum_{i=1}^{n}i = \frac{n(n+1)}{2}
\]

Now we might have counted too many numbers because $p$ itself might be included if it is divisible by $3$ or $5$.
The full solution can then be computed in constant time:

\begin{figure}[H]
\centering
\lstinputlisting[language=c, firstline=22, lastline=25]{\sol{001}{c}/solution.c}
\caption{A constant-time solution}
\end{figure}


\end{document}
