\documentclass[main.tex]{subfiles}
\begin{document}

\begin{prob}{1}{Multiples of 3 and 5}
  \begin{problem}
    If we list all the natural numbers below 10 that are multiples of 3 or 5, we get 3, 5, 6 and 9. The sum of these multiples is 23.
    Find the sum of all the multiples of 3 or 5 below 1000.
  \end{problem}
  \begin{solutions}
    \begin{solution}{A naive solution}
      An input size of $p=1000$ is small enough to let us get away with a naive solution.
      The naive solution, to go over every number up to $p$ and adding up the ones that are divisible by $3$ or by $5$, gives us an $O(p)$ runtime.

      \begin{figure}[H]
        \centering
        \inputminted[firstline=14, lastline=24]{c}{\sol{001}{c}/naive.c}
        \caption{A naive solution in C}
      \end{figure}
      \complexity{n}{1}
    \end{solution}

    \begin{solution}{A better solution}
      As it turns out, there's a better way to solve this problem.
      Because the numbers that are divisible by $3$ and $5$ are easily predicitible, we can calculate the solution with the following formula:

      \[
        \sum_{i=0}^{3i < p}3i
        + \sum_{i=0}^{5i < p}5i
        - \sum_{i=0}^{15i < p}15i
      \]

      This formula can quite literally be implemented so find the solution.
      In this solution, we check less than half of the numbers we did in the last solution.

      \begin{figure}[H]
        \centering
        \inputminted[firstline=15, lastline=28]{c}{\sol{001}{c}/better.c}
        \caption{A better solution in C}
      \end{figure}

      The same can be done in haskell on two lines:
      \begin{figure}[H]
        \centering
        \inputminted[firstline=7, lastline=10]{haskell}{\sol{001}{haskell}/better.hs}
        \caption{A better solution in Haskell}
      \end{figure}
      \complexity{n}{1}
    \end{solution}

    \begin{solution}{A constant-time solution}
      Looking at the above formula, we find that there's a constant-time algorithm to calculate each term and therefore the entire expression.\algref{multiples-under}
      \begin{figure}[H]
        \centering
        \inputminted[firstline=16, lastline=20]{c}{\sol{001}{c}/solution.c}
        \caption{A better solution in Haskell}
      \end{figure}
      \complexity{1}{1}
    \end{solution}
  \end{solutions}
\end{prob}


\end{document}
