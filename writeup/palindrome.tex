\documentclass[main.tex]{subfiles}
\begin{document}

\chapter{Palindromes}

\begin{de}
  \label{de:palindrome}
  An $n$-palindrome is a natural number that reads the same both forward and backward when written in base $n$.
  `palindrome' is also used as a shorthand for `$10$-palindrome`.
\end{de}

\begin{th}
  \label{th:number-of-p-digit-n-palindromes}
  The number of $p$-digit $n$-palindromes is $(n-1)\cdot n^{p-1}$ if $p$ is even and $(n-1)\cdot n^{p}$ if $p$ is odd.
  \TODO{prove the number of $p$-digit $n$-palindromes}
  % http://math.stackexchange.com/questions/287582/how-many-n-digit-palindromes-are-there
\end{th}

\end{document}
