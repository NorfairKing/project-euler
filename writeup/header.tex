\theoremstyle{plain}
\newtheorem{thm}{Theorem}[chapter] %Reset counter every chapter
\theoremstyle{definition}
\newmdtheoremenv{de}[thm]{Definition} % Definitie with frame
\newtheorem{prop}[thm]{Property}
\let\th\undefined
\newtheorem{th}[thm]{Theorem}
\newtheorem{pr}[thm]{Proposition}
\newtheorem{nt}[thm]{Note}
\newtheorem{ex}[thm]{Example}
\newtheorem{cex}[thm]{Counterexample}
\newtheorem{lem}[thm]{Lemma}
\newtheorem{alg}[thm]{Algorithm}

% Nicer TODO's
\newcommand{\TODO}[1]{\todo[color=red,inline,size=\small]{TODO: #1}}
\newcommand{\extra}[1]{\todo[color=orange,inline,size=\small]{EXTRA: #1}}
\newcommand{\clarify}[1]{\todo[color=yellow,inline,size=\small]{CLARIFY: #1}}
\newcommand{\question}[1]{\todo[color=green,inline,size=\small]{QUESTION: #1}}

\newcommand{\why}[0]{\clarify{why?}}
\newcommand{\needed}[0]{\clarify{reference?}}

% More space between math arrays
\renewcommand{\arraystretch}{1.25}

% An actual QED symbol.
\renewcommand{\qedsymbol}{$\square$}

\definecolor{solarback}{HTML}{FDF6E3}
\definecolor{solarfront}{HTML}{657A81}
\mdfdefinestyle{klad}{linewidth=0pt,backgroundcolor=solarback,fontcolor=solarfront}
\newenvironment{klad}{\begin{mdframed}[style=klad]}{\end{mdframed}}

\newenvironment{problem}{}{}

% \nointent everywhere
\setlength\parindent{0pt}

% Eden root
\newcommand{\eden}[0]{..}
\newcommand{\prob}[1]{\eden/#1}
\newcommand{\sol}[2]{\prob{#1}/#2}

\lstset{
  basicstyle=\small
}
