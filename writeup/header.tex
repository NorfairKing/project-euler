\theoremstyle{plain}
\newtheorem{thm}{Theorem}[chapter] %Reset counter every chapter
\theoremstyle{definition}
\newmdtheoremenv{de}[thm]{Definition} % Definitie with frame
\newtheorem{prop}[thm]{Property}
\let\th\undefined
\newtheorem{th}[thm]{Theorem}
\newtheorem{pr}[thm]{Proposition}
\newtheorem{nt}[thm]{Note}
\newtheorem{ex}[thm]{Example}
\newtheorem{cex}[thm]{Counterexample}
\newtheorem{lem}[thm]{Lemma}
\newmdtheoremenv{alg}[thm]{Algorithm}

\newenvironment{algorithm}[3]
{\begin{alg}
    \label{alg:#3}
    Time: $O(#1)$, Space: $O(#2)$\\
  }
{\end{alg}}

\newenvironment{algorithm-description}[0]
{}
{}
\newenvironment{algorithm-explanation}[0]
{}
{}

\newcommand{\algref}[1]{\footnote{See definition \ref{de:#1} on page \pageref{alg:#1}.}}
\newcommand{\deref}[1]{\footnote{See definition \ref{de:#1} on page \pageref{de:#1}.}}
\newcommand{\prref}[1]{\footnote{See proposition \ref{pr:#1} on page \pageref{pr:#1}.}}
\newcommand{\propref}[1]{\footnote{See property \ref{prop:#1} on page \pageref{prop:#1}.}}


% Nicer TODO's
\newcommand{\TODO}[1]{\todo[color=red,inline,size=\small]{TODO: #1}}
\newcommand{\extra}[1]{\todo[color=orange,inline,size=\small]{EXTRA: #1}}
\newcommand{\clarify}[1]{\todo[color=yellow,inline,size=\small]{CLARIFY: #1}}
\newcommand{\question}[1]{\todo[color=green,inline,size=\small]{QUESTION: #1}}

\newcommand{\why}[0]{\clarify{why?}}
\newcommand{\needed}[0]{\clarify{reference?}}

\newcommand{\term}[1]{\index{#1}\textbf{#1}}

% More space between math arrays
\renewcommand{\arraystretch}{1.25}

% An actual QED symbol.
\renewcommand{\qedsymbol}{$\square$}

\definecolor{solarback}{HTML}{FDF6E3}
\definecolor{solarfront}{HTML}{657A81}
\mdfdefinestyle{klad}{linewidth=0pt,backgroundcolor=solarback,fontcolor=solarfront}
\newenvironment{klad}{\begin{mdframed}[style=klad]}{\end{mdframed}}

\newenvironment{prob}[2]
{
\newpage
\section{Problem #1: #2}
}
{}

\newenvironment{problem}
{}
{}

\newenvironment{solutions}
{}
{}

\newenvironment{solution}[1]
{
\subsection{#1}
}
{}

\newenvironment{libraryfile}
{}
{}

% \nointent everywhere
\setlength\parindent{0pt}

% Eden root
\newcommand{\eden}[0]{..}
\newcommand{\sol}[2]{\eden/#1/#2}
\newcommand{\lib}[2]{\eden/lib/#1/#2}

\lstset{
  basicstyle=\small\ttfamily
}


\makeatletter

% Partial inclusions
\newread\pin@file
\newcounter{pinlineno}
\newcommand\pin@accu{}
\newcommand\pin@ext{pintmp}
% inputs #3, selecting only lines #1 to #2 (inclusive)
\newcommand*\partialinput [3] {%
  \IfFileExists{#3}{%
    \openin\pin@file #3
    % skip lines 1 to #1 (exclusive)
    \setcounter{pinlineno}{1}
    \@whilenum\value{pinlineno}<#1 \do{%
      \read\pin@file to\pin@line
      \stepcounter{pinlineno}%
    }
    % prepare reading lines #1 to #2 inclusive
    \addtocounter{pinlineno}{-1}
    \let\pin@accu\empty
    \begingroup
    \endlinechar\newlinechar
    \@whilenum\value{pinlineno}<#2 \do{%
      % use safe catcodes provided by e-TeX's \readline
      \readline\pin@file to\pin@line
      \edef\pin@accu{\pin@accu\pin@line}%
      \stepcounter{pinlineno}%
    }
    \closein\pin@file
    \expandafter\endgroup
    \scantokens\expandafter{\pin@accu}%
  }{%
    \errmessage{File `#3' doesn't exist!}%
  }%
}

% Better boxes
\makeatletter
\renewcommand{\boxed}[1]{\text{\fboxsep=.2em\fbox{\m@th$\displaystyle#1$}}}
\makeatother

% Leftarrow and Rightarrow with boxes around them 
\newcommand{\bla}{$\boxed{\Leftarrow}$\ } % Boxed left arrow
\newcommand{\bra}{$\boxed{\Rightarrow}$\ } % Boxed right arrow

\newcommand{\complexity}[2]{\begin{mdframed}Time: $O(#1)$, Space: $O(#2)$\end{mdframed}}

\makeatother

\newcommand{\mand}{\ \wedge\ } % math and
\newcommand{\tand}{\ \text{ and }\ } %text and
\newcommand{\mor}{\ \vee\ } % math of
\newcommand{\tor}{\ \text{ or }\ } % text or
\newcommand{\mdiv}{\mid} % math divides
\newcommand{\mimplies}{\Rightarrow} % math implies
