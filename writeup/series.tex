\documentclass[main.tex]{subfiles}
\begin{document}
\chapter{Series}

\begin{de}
  A sequence $(a_{n})_{n}$ of elements in a set $A$ is a function $a: N
  \rightarrow A: n \rightarrow a_{n}$ that identifies every natural number $n$ with an element of
  the set $A$. The element $a_{n}$ in the sequence $(a_{n})_{n}$ is called a the
  $n$-th element of that sequence.
\end{de}

\begin{de}
  Let $(a_{n})_{n}$ be a sequence.
  A sequence $(s_{n})_{n}$ as follows is the series of the sequence $(a_{n})_{n}$.
  \[ s_{n} = \sum_{i=0}^{n}a_{i} \]
\end{de}

\begin{de}
  We often use $\sum_{n}a_n$ as a shorthand for $\left(\sum_{i=0}^{n}a_{i}\right)_{n}$.
\end{de}

\begin{pr}
  \label{pr:sum-of-first-n-natural-numbers}
  \[
    \sum_{i=1}^n i = \frac{n\cdot(n+1)}{2}
  \]
  \begin{proof}
    \begin{align*}
      \sum_{i=1}^n i
      &= 1 + 2 + 3 + \dotsb + (n-2) + (n-1) + n\\
      &= (1 + n) + (2+n-1) + (3+n-2) + \dotsb + (i+n-i+1) + \dotsb\\
      &= \frac{n}{2} (n+1)
      &= \frac{n\cdot(n+1)}{2}
    \end{align*}
  \end{proof}
\end{pr}


\section{Geometric series}

\begin{de}
  A geometric series is a series of the following form:
  \[ \sum_{i}ar^{i} \]
\end{de}

\begin{pr}
  \label{pr:closed-form-expression-nth-term-of-geometric-series}
  Let $\sum_{i}ar^{i}$ be a geometric series.
  \[ \sum_{i=0}^{n-1}ar^{i} = a\frac{1 - r^{n}}{1 - r} \]
  \TODO{proof of the closed form expression of the nth term of a geometric series}
\end{pr}


\section{The Fibonacci sequence}

\begin{de}
  The \term{Fibonacci sequence} $(F_{n})_{n}$ is recursively defined as follows:
  \[ F_{0} = 1,\ F_{1} = 1,\ F_{n} = F_{n-1} + F_{n-2} \]
\end{de}

\begin{prop}
  \label{prop:every-third-fibonacci-term-even}
  Every third Fibonacci number is even, that is, for every $k$, $F_{3k+1}$ is even.
  \begin{proof}             
    \noindent             
    \begin{itemize}       
      \item Base case: $F_{1} = 0$ is even.
      \item Inductive step: If $F_{3k+1}$ is even, then $F_{3k+4}$ is even.\\
        \[
          F_{3k+4} = F_{3k+3} + F_{3k+2} = 2F_{3k+2} + F_{3k+1}
        \]
        Because both $F_{3k+1}$ and $2F_{3k+2}$ are even, $F_{3k+4}$ must be even.
    \end{itemize}
  \end{proof}
\end{prop}

\begin{de}
  The \term{golden ratio} $\phi$ is defined as follows:
  \[ \phi = \frac{1 + \sqrt{5}}{2} \]
\end{de}

\begin{prop}
  \[ \lim_{n \rightarrow +\infty}F_{n} = \frac{1 + \sqrt{5}}{2} \]
  \TODO{Proof of limit of ratio of terms of Fibonacci sequence}
\end{prop}

\begin{prop}
  \label{prop:closed-form-expression-fibonacci-term}
  \[ F_{n} = \frac{\phi^{n} - (-\phi)^{-n}}{\sqrt{5}} \]
  \TODO{Proof of closed form expression of term of Fibonacci sequence}
\end{prop}


\end{document}
