\documentclass[main.tex]{subfiles}
\begin{document}

\begin{prob}{6}{Sum square difference}
    \begin{problem}
        The sum of the squares of the first ten natural numbers is,
        \[ 1^2 + 2^2 + ... + 10^2 = 385 \]

        The square of the sum of the first ten natural numbers is,
        \[ (1 + 2 + ... + 10)^2 = 55^2 = 3025 \]

        Hence the difference between the sum of the squares of the first ten natural numbers and the square of the sum is $3025 - 385 = 2640$.

        Find the difference between the sum of the squares of the first one hundred natural numbers and the square of the sum.
    \end{problem}
    \begin{solutions}
        \begin{solution}{A Naive solution}
            Let $p$ be $100$.

            The naive solution is of course to just calculate both the sum of squares and the square of the sum.

            \begin{figure}[H]
                \lstinputlisting[language=haskell, firstline=1, lastline=10]{\sol{006}{haskell}/naive.hs}
                \caption{A naive solution}
            \end{figure}

            This is an $O(p)$ solution and it's fast enough to reach a solution in under one minute.
        \end{solution}

        \begin{solution}{A better solution}
            There is, of course, a better solution.
            The question even alludes to it.

            The sum of natural numbers up to $p$ can be obtained in constant time: \needed
            \[ \sum_{i}i = \frac{n(n+1)}{2} \]

            The sum of the first $p$ natural squares can be obtained in constant time as well: \why
            \[ \sum_{i}i^{2} = \frac{n(n+1)(n+2)}{6} \]

            This means that the solution can be computed in constant time.

            \begin{figure}[H]
                \lstinputlisting[language=haskell, firstline=1, lastline=2]{\sol{006}{haskell}/solution.hs}
                \caption{A better solution}
            \end{figure}
        \end{solution}
    \end{solutions}
\end{prob}

\end{document}
