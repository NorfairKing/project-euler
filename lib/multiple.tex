\documentclass[main.tex]{subfiles}
\begin{document}

\chapter{Multiple}

\begin{libraryfile}
  \begin{algorithm}{1}{1}{multiples-under}
    \begin{algorithm-description}
        Compute the sum of all multiples of a given number $n$ below a given number $p$.
    \end{algorithm-description}
    \begin{algorithm-explanation}
      The exact solution would be the value of the following expression:
      \[ \sum_{i=0}^{ni < p}ni \]
      If we would want all the multiples up to \textit{and including} $p$, we could rewrite the solution as follows:
      \[ \sum_{i=0}^{i \le \floor*{\frac{p}{n}}}ni \]
      If we factor out $n$, we see the sum of the first $\floor*{\frac{p}{n}}$ appearing and that has a closed form expression.\prref{sum-of-first-n-natural-numbers}
      That means the above is equal to the following:
      \[ n\frac{\floor{\frac{p}{n}}\cdot \left(\floor{\frac{p}{n}} + 1\right)}{2} \]
      Now we might have counted too many numbers because $p$ itself will be included if it is divisible by $n$.
      Because we've considered numbers up to and including $p$, we will have included $p$ if $p$ is divisible by $n$ so we have to substract $p$ in that case to get the result.

      \begin{figure}[H]
        \centering
        \lstinputlisting[language=c, firstline=3, lastline=6] {\lib{c}{multiple.c}}
        \caption{An implementation in C}
      \end{figure}
    \end{algorithm-explanation}
  \end{algorithm}
\end{libraryfile}

\end{document}
