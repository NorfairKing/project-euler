\documentclass[main.tex]{subfiles}
\begin{document}

\chapter{Order}

\begin{libraryfile}
  \begin{algorithm}{1}{1}{min-max}
    \begin{algorithm-description}
        Compute the minimum/maximum of two given numbers.
    \end{algorithm-description}
    \begin{algorithm-explanation}
    Not all functions are hard to write.
    Compared to the mathematical equivalent\deref{inequality-natural-numbers} and thanks to the way software works on top of hardware, the maximum and minimum functions are quite easy to define:
    \begin{figure}[H]
        \centering
        \lstinputlisting[language=c, firstline=4, lastline=4] {\lib{c}{order.h}}
        \vspace{-10px}
        \lstinputlisting[language=c, firstline=3, lastline=3] {\lib{c}{order.c}}
        \lstinputlisting[language=c, firstline=7, lastline=7]{\lib{c}{order.h}}
        \vspace{-10px}
        \lstinputlisting[language=c, firstline=5, lastline=5] {\lib{c}{order.c}}
        \caption{The maximum and minimum functions in C}
    \end{figure}
  \end{algorithm-explanation}
  \end{algorithm}
\end{libraryfile}

\end{document}
