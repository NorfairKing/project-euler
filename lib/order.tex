\documentclass[main.tex]{subfiles}
\begin{document}

\chapter{Order}

\begin{libraryfile}
  \begin{algorithm}{1}{1}{min-max}
    \begin{algorithm-description}
        Compute the minimum/maximum of two given numbers.
    \end{algorithm-description}
    \begin{algorithm-explanation}
    Not all functions are hard to write.
    Compared to the mathematical equivalent\deref{inequality-natural-numbers} and thanks to the way software works on top of hardware, the maximum and minimum functions are quite easy to define:
    \begin{figure}[H]
        \centering
        \inputminted[firstline=4, lastline=4]{c}{\lib{c}{order.h}}
        \vspace{-27px}
        \inputminted[firstline=4, lastline=4]{c}{\lib{c}{order.c}}
        \inputminted[firstline=7, lastline=7]{c}{\lib{c}{order.h}}
        \vspace{-27px}
        \inputminted[firstline=6, lastline=6]{c}{\lib{c}{order.c}}
        \caption{The maximum and minimum functions in C}
    \end{figure}
  \end{algorithm-explanation}
  \end{algorithm}
\end{libraryfile}

\end{document}
