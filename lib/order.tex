\documentclass[main.tex]{subfiles}
\begin{document}

\chapter{Order}

\begin{libraryfile}
    Not all functions are hard to write.
    Compared to the mathematical equivalent\needed and thanks to the way software works on top of hardware, the maximum and minimum functions are quite easy to define:
    \begin{figure}[H]
        \centering
        \lstinputlisting[language=c, firstline=4, lastline=4] {\lib{c}{order.h}}
        \vspace{-10px}
        \lstinputlisting[language=c, firstline=3, lastline=3] {\lib{c}{order.c}}
        \lstinputlisting[language=c, firstline=7, lastline=7]{\lib{c}{order.h}}
        \vspace{-10px}
        \lstinputlisting[language=c, firstline=5, lastline=5] {\lib{c}{order.c}}
        \caption{The maximum and minimum functions}
    \end{figure}

\end{libraryfile}

\end{document}
