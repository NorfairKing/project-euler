\documentclass[main.tex]{subfiles}
\begin{document}

\begin{prob}{5}{Smallest multiple}
    \begin{problem}
        2520 is the smallest number that can be divided by each of the numbers from 1 to 10 without any remainder.
        What is the smallest positive number that is evenly divisible by all of the numbers from 1 to 20?
    \end{problem}
    \begin{solutions}
        \begin{solution}{A Naive solution}
            Let $p$ be $20$.

            A naive solution would be to start from $1$ and check every number for divisibility by every number up to $p$.
            \begin{figure}[H]
                \lstinputlisting[language=c, firstline=16, lastline=33]{\sol{005}{c}/naive.c}
                \caption{A naive solution}
            \end{figure}
            Even with $p=20$, this runs in under one minute.
            This solution runs in $O(pN)$ where $N$ is the result.
        \end{solution}
        \begin{solution}{A better solution}
            If we can find out what the prime factors are of $p$ then we can construct $p$ by multiplying them all together.
            Let's start from $p!$.
            We know that $p!$ is divisible by every number up to and including $p$.
            There are factors that can be left out though.
            Take $p=4$ for example. $4! = 4 \cdot 3 \cdot 2 = 24$ and $24$ is divisible by $2$, $3$ and $4$ but $12$ is also divisible by $2$, $3$ and $4$.
            The reason is that is the case is that the prime decomposition $2^{2}$ of $4$ contains one of the factors: $2$.
            All we need to do to construct the solution is find all prime \textbf{unique} multiplicities of all prime factors of the numbers up to $p$.
            In the example, the prime $2$ has multiplicity $1$ in de prime decomposition of $2$ and multiplicity $2$ in the prime decomposition of $4$ so the solution for $p=4$ will have $2^{2}$ in its prime decomposition.

            First we precompute the primes up to and including $p$.
            \lstinputlisting[language=c, firstline=17, lastline=22]{\sol{005}{c}/solution.c}
            Then we aggregate all the multiplicities of the prime decompositions of the numbers up to and including $p$.
            \lstinputlisting[language=c, firstline=24, lastline=35]{\sol{005}{c}/solution.c}
            Lastly we multiply all the prime factors together with their respective multiplicity.
            \lstinputlisting[language=c, firstline=37, lastline=51]{\sol{005}{c}/solution.c}
            Calculating all the primes untill $p+1$ takes $O(p\sqrt{p})$ time.
            Calculating all mutiplicites takes $O(p\sqrt{p})$ time again.
            Multiplying all the prime factors together takes time proportional to the amount of prime factors of the result.
            This is less than $O(p\sqrt{p})$ \why so this solution runs in $O(p\sqrt{p})$.
        \end{solution}
    \end{solutions}
\end{prob}

\end{document}
