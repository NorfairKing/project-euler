\documentclass[main.tex]{subfiles}
\begin{document}

\subsection{Problem 3: Largest Prime Factor}
\begin{problem}
The prime factors of 13195 are 5, 7, 13 and 29.
What is the largest prime factor of the number 600851475143 ?
\end{problem}

\subsubsection{A Naive solution}
Let $p$ be $600851475143$.
A very naive solution would be to go over every number from $1$ until $600851475143$, check if it's prime and if it's a factor, and keep the largest.
\begin{figure}[H]
    \lstinputlisting[language=c, firstline=16, lastline=25]{\sol{003}{c}/naive.c}
    \caption{A Naïve solution}
\end{figure}
This solution is $O(p)$, but with $p$ as big as it is, this will not finish within one minute.

\subsubsection{A sligtly better solution}
A better solution would be to start from $p$ and work our way downwards until we find a prime factor.
Because we're now working out way down, it's a good idea to consider possible upper bounds for the largest prime factors of $p$.
$p$ is an upper bound, but $\nicefrac{p}{2}$ is also an upper bound. \why
\begin{figure}[H]
    \lstinputlisting[language=c, firstline=16, lastline=24]{\sol{003}{c}/better.c}
    \caption{A sligtly better solution}
\end{figure}
This is still $O(p)$ though.

\subsubsection{An even better solution}
There is an easier way to know when you can stop looking for larger prime factors.
Because the prime factorisation of a number is unique, the product of all prime factors (with their respective multiplicity) is $p$ itself.
If we divide out every prime factor we find starting from $1$, the last prime factor we found will be the largest prime factor of $p$.

\begin{figure}[H]
    \lstinputlisting[language=c, firstline=16, lastline=27]{\sol{003}{c}/solution.c}
    \caption{A even better solution}
\end{figure}






\end{document}
