\documentclass[main.tex]{subfiles}
\begin{document}

\begin{prob}{3}{Largest Prime Factor}
  \begin{problem}
    The prime factors of 13195 are 5, 7, 13 and 29.
    What is the largest prime factor of the number 600851475143?
  \end{problem}

  \begin{solutions}
    \begin{solution}{A Naive solution}
      Let $p$ be $600851475143$.
      A very naive solution would be to go over every number from $1$ until $p$, check if it's prime and if it's a factor, and keep the largest.
      \begin{figure}[H]
        \inputminted[firstline=19, lastline=28]{c}{\sol{003}{c}/naive.c}
        \caption{A Naive solution in C}
      \end{figure}
      This solution is $O(p)$, but with $p$ as big as it is, this will not finish within one minute.
      \complexity{p}{1}
    \end{solution}

    \begin{solution}{A slightly better solution}
      A better solution would be to start from $p$ and work our way downwards until we find a prime factor.
      Working downwards allows us to stop as soon as we find a prime factor.
      \begin{figure}[H]
        \inputminted[firstline=19, lastline=27]{c}{\sol{003}{c}/better.c}
        \caption{A slightly better solution in C}
      \end{figure}
      This solution works best (in $\sqrt(p)$ time) if $p$ is a prime.
      \complexity{p}{1}
    \end{solution}

    \begin{solution}{A more better solution}
      There is an easier way to know when you can stop looking for larger prime factors.
      Because the prime factorisation of a number is unique\prref{prime-factorisation-unique}, the product of all prime factors (with their respective multiplicity) is $p$ itself.
      If we divide out every prime factor we find starting from $1$, the last prime factor we found will be the largest prime factor of $p$.

      \begin{figure}[H]
        \inputminted[firstline=17, lastline=28]{c++}{\sol{003}{c++}/solution.cc}
        \caption{A better solution in C++}
      \end{figure}
      \begin{figure}[H]
        \inputminted[firstline=6, lastline=7]{haskell}{\sol{003}{haskell}/solution.hs}
        \caption{A more elegant version in Haskell}
      \end{figure}
      This solution works best if $p$ has a lot of prime factors.
      \complexity{\sqrt{p}}{1}
    \end{solution}
  \end{solutions}
\end{prob}

\end{document}
