\documentclass[main.tex]{subfiles}
\begin{document}

\begin{prob}{4}{Largest Palindrome Product}
    \begin{problem}
        A palindromic number reads the same both ways. The largest palindrome made from the product of two 2-digit numbers is 9009 = 91 × 99.
        Find the largest palindrome made from the product of two 3-digit numbers.
    \end{problem}
    \begin{solutions}
        \begin{solution}{A Naive solution}
            Let $p$ be $3$.
            The most naive solution would be to go over every pair of $p$-digit numbers and keep the largest palindrome product.
            \begin{figure}[H]
                \lstinputlisting[language=c, firstline=15, lastline=28]{\sol{004}{c}/naive.c}
                \caption{A Naïve solution}
            \end{figure}
            This solution is $O(N^2)$ in the amount of $p$-digit numbers, so $O(10^{2p})$.
            Given that $p$ is only $3$, this solution is already good enough to get an answer within one minute.
        \end{solution}
        \TODO{ A solution using a priority queue to look downwards }

        \begin{solution}{A better solution}
            The worst part of the naive solution is that we're still checking a lot of number that aren't even palindromes.
            We could generate palindromes, check whether they have a $p$-digit divisor with a $p$-digit quotient and keep the largest.
            We could also generate palindromes starting from the largest and then we could stop as soon as we find one that is divisble by a $p$-digit number with a $p$-digit qutient.

            To generate palindromes from the top down, we first generate the left half of the number, copy the left half to the right and at the same time reverse the order.

            From these digits, we construct the represented palindrome and check whether it's divisble by a $p$-digit number with a $p$-digit quotient.

            \begin{figure}[H]
                \lstinputlisting[language=c, firstline=17, lastline=44]{\sol{004}{c}/solution.c}
                \caption{A sligtly better solution}
            \end{figure}

            \TODO{What's the runtime complexity of this algorithm?}
        \end{solution}
    \end{solutions}
\end{prob}



\end{document}
