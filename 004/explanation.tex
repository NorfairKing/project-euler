\documentclass[main.tex]{subfiles}
\begin{document}

\begin{prob}{4}{Largest Palindrome Product}
  \begin{problem}
    A palindromic number reads the same both ways.
    The largest palindrome made from the product of two $2$-digit numbers is $9009$ = $91 \times 99$.
    Find the largest palindrome made from the product of two $3$-digit numbers.
  \end{problem}
  \begin{solutions}
    \begin{solution}{A Naive solution}
      Let $p$ be $3$.
      The most naive solution would be to go over every pair of $p$-digit numbers and keep the largest palindrome product.
      \begin{figure}[H]
        \inputminted[firstline=16, lastline=26]{c++}{\sol{004}{c++}/naive.cc}
        \caption{A naive solution in C++}
      \end{figure}
      This solution is $O(N^2)$ in the amount of $p$-digit numbers, so $O(10^{2p})$.
      Given that $p$ is only $3$, this solution is already good enough to get an answer within one minute.
      \complexity{10^{2p}}{p}
    \end{solution}

    \begin{solution}{A better solution}
      The worst part of the naive solution is that we're still checking a lot of number that aren't even palindromes.
      We could generate palindromes, check whether they have a $p$-digit divisor with a $p$-digit quotient and keep the largest.
      There are either $9\cdot 10^{p-1}$ or $9 \cdot 10^{p}$ palindromic $p$ digit numbers depending on whether $p$ is even or odd. \thref{number-of-p-digit-n-palindromes}
      This means that we're checking a lot less numbers.
      We could also generate palindromes starting from the largest and then we could stop as soon as we find one that is divisble by a $p$-digit number with a $p$-digit qutient.

      We can generate the palindromes' digits in $O(\log(p))$ time.
      From these digits, we construct the represented palindrome and check whether it's divisble by a $p$-digit number with a $p$-digit quotient.

      \begin{figure}[H]
        \inputminted[firstline=18, lastline=41]{c++}{\sol{004}{c++}/solution.cc}
        \caption{A better solution in C++}
      \end{figure}

      Checking whether a palindrom is divisible by a $p$-digit number is done by going from $hi$ to $lo$ with trial division.
      This sounds like it has a bad runtime complexity ($O(10^{p})$)but because the result palindrome will be fairly close to $hi$, this is actually very fast.
      \complexity{10^{2p}}{p}
    \end{solution}
    \question{Is there a way to enumerate all $p$-digit numbers in reverse order? Then there would be an easier way.}
  \end{solutions}
\end{prob}



\end{document}
